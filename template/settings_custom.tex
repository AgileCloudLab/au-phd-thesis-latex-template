
\usepackage[]{nomencl} %Create nmoenclature
\makenomenclature




% Allow the thesis to have bib in each chapter
\usepackage[sectionbib]{chapterbib}

% Hide the section in toc
\newcommand{\nocontentsline}[3]{}
\newcommand{\tocless}[2]{\bgroup\let\addcontentsline=\nocontentsline#1{#2}\egroup}

%\usepackage{bitfoot} %make wonders to 
\usepackage{catchfilebetweentags}
\usepackage{wrapfig}

\usepackage{adjustbox} %Makes it possible to scale tables, if they are too big
\usepackage[final]{pdfpages}

\newcommand{\csubfloat}[2][]{%
  \makebox[0pt]{\subfloat[#1]{#2}}%
}

\usepackage{placeins} % Makes it possible to use floatbarriers = all declaired float before this point has to be printed

\usepackage{rotating} %Make it possible to rotate tables etc.
\usepackage{lscape}
\usepackage{bibentry} %Print bib entry in txt




\usepackage{afterpage} %Make it possible to place figure on odd/even page

\usepackage{multicol}

\usepackage{float}  % For using "H" to position figures

\usepackage{enumitem} % Make custom enumerations

\usepackage{epigraph} % To make quotes (https://tex.stackexchange.com/questions/53377/inspirational-quote-at-start-of-chapter)

\usepackage{csquotes} % To make different types of quotes (https://da.sharelatex.com/learn/Typesetting_quotations)



%\usepackage[a4,pdflatex,center, cam]{crop}

% For equations:
\usepackage{dsfont}								% For math symbols
\DeclareMathAlphabet{\mathsfit}{\encodingdefault}{\sfdefault}{m}{sl}
\SetMathAlphabet{\mathsfit}{bold}{\encodingdefault}{\sfdefault}{bx}{n}
\newcommand{\tens}[1]{\bm{\mathsfit{#1}}}
\def\tA{{\tens{A}}}
\newcommand\numberthis{\addtocounter{equation}{1}\tag{\theequation}}  % To number equations written using \begin{align*}

% For theorem:
\newtheorem{theorem}{Theorem}
\newtheorem{corollary}{Corollary}
% \newenvironment{proof}{\begin{IEEEproof}}{\end{IEEEproof}}


\makenoidxglossaries
%\input{template/glossaries.tex}

\usepackage{lipsum}

%% Matlab2tikz
\usepackage{pgfplots}
  \pgfplotsset{width=.83\textwidth,compat=newest}
  %% the following commands are needed for some matlab2tikz features
  \usetikzlibrary{plotmarks}
  \usetikzlibrary{arrows.meta}
  \usepgfplotslibrary{patchplots}
  \usepackage{grffile}
  
\newenvironment{horizontallegend}[1][]{%
    \begingroup
    % inits/clears the lists (which might be populated from previous
    % axes):
    \pgfplots@init@cleared@structures
    \pgfplotsset{#1}%
}{%
    % draws the legend:
    \pgfplots@createlegend
    \endgroup
}%


%% Collection of papers
%\usepackage{collect}
%\definecollection{papers}
%\newcounter{PapersCounter}
%\makeatletter
%\newenvironment{paper}[3]
%  {\@nameuse{collect}{papers}{}
%    {}
%    {\clearpage \stepcounter{PapersCounter} \chapter*{Paper \thePapersCounter}\addcontentsline{toc}{section}{Paper \thePapersCounter : #1}\textbf{#1}\\\textit{#2}\\ \vspace{10 mm}\\#3\newpage}
%    {}%
%  }
%{\@nameuse{endcollect}}
%\makeatother

\usepackage{bookmark}
\usepackage{fontawesome5} % font awesome                                                                                                            
                                                                                                            \usepackage{textcomp}
                                                                                                            \usepackage{pifont}

                                                                                                            \usepackage{adjustbox}

% Collection of papers
\usepackage{collect}
\definecollection{papers}
\newcounter{PapersCounter}
\makeatletter
\newenvironment{paper}[6]
{\@nameuse{collect}{papers}{}
    {}
    {\clearpage \stepcounter{PapersCounter} \paperlabel{#1}{\thePapersCounter} \chapter*{Paper \thePapersCounter}\addcontentsline{toc}{section}{Paper \thePapersCounter : #2}\textbf{#2}\\\textit{#3}\\ \vspace{10 mm}\\#4 \vspace{10 mm}\\#5 \vspace{0 mm}\\\textit{#6} \newpage}
    {}%
}
{\@nameuse{endcollect}}
\makeatother

%

% Paper referencing
\makeatletter
\newcommand{\paperlabel}[2]{%
    \protected@write \@auxout {}{\string \newlabel {#1}{{Paper #2}{\thepage}{#2}{#1}{}} }%
    \hypertarget{#1}{}
}
\makeatother

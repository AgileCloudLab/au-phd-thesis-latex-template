%%%%%%%%%%%%%%%%%%%%%%%%%%%%%%%%%%%%%%%%%%%%%%
%
%		Thesis Settings
%		Custom settings
%
%		2011
%
%%%%%%%%%%%%%%%%%%%%%%%%%%%%%%%%%%%%%%%%%%%%%%

%
%   Use this file for your own custom packages, command-definitions, etc...
%

% the following lines are for creating a simplified TO-DO box. However since boites is not per default installed with all latex-distributions, we have removed this example again
% if you want to use it and do not have "boites" installed, you can get it from here: http://www.ctan.org/tex-archive/macros/latex/contrib/boites
%
%\usepackage{boites,boites_exemples}
%\newcommand{\todolist}[1]{\begin{boiteepaisseavecuntitre}{TO DO in this chapter} #1 \end{boiteepaisseavecuntitre}}  % creates a little box
% %\newcommand{\todolist}[1]{}  % to be used when to do is not to be printed
\usepackage{makeidx}
\makeindex

\usepackage[]{nomencl} %Create nmoenclature
\makenomenclature

\usepackage{nameref}
\makeatletter
\newcommand*{\currentname}{\@currentlabelname}
\makeatother

\usepackage{parskip}




% \setlength{\parskip}{0em}

% Reference to chapter and part names
% See: https://tex.stackexchange.com/questions/62241/how-to-get-the-current-chapter-name-section-name-subsection-name-etc
\let\Partmark\partmark
\def\partmark#1{\def\Partname{#1}\Partmark{#1}}
\let\Chaptermark\chaptermark
\def\chaptermark#1{\def\Chaptername{#1}\Chaptermark{#1}}
\let\Sectionmark\sectionmark
\def\sectionmark#1{\def\Sectionname{#1}\Sectionmark{#1}}
\let\Subsectionmark\subsectionmark
\def\subsectionmark#1{\def\Subsectionname{#1}\Subsectionmark{#1}}
\let\Subsubsectionmark\subsubsectionmark
\def\subsubsectionmark#1{\def\Subsubsectionname{#1}\Subsubsectionmark{#1}}

\usepackage[colorinlistoftodos,textwidth={\thesisPaperMarginInner - 2mm}]{todonotes}
% \usepackage[colorinlistoftodos,textwidth={\thesisPaperMarginInner - 2mm},disable]{todonotes}
\newcommand{\todonote}[2]{
	\todo[inline, color=cyan, caption={\Partname~$>$ \Chaptername~$>$  #1}]{
		\begin{minipage}[t]{\linewidth}
			\textbf{#1}:
			#2
		\end{minipage}
	}
}
\newcommand{\todonotes}[2]{\todonote{#1}{#2}}
%\newcommand{\note}[2]{#1 : #2}
\newcommand{\comment}[1]{\todo[size=\scriptsize]{#1}}
\newcommand{\todoref}[1]{\todo[color=green, size=\tiny, caption={\Partname~$>$ \Chaptername~$>$ #1}]{Ref: #1}}
\newcommand{\tbc}{\todo[inline, color=yellow, caption={\Partname~$>$ \Chaptername~$>$ To be continued}]{To be continued...}}
\newcommand{\TBC}{\tbc}
\newcommand{\pageplaceholder}[1]{\todo[inline, color=red, caption={Placeholder: #1}]{Start of placehoder \\ \vspace{5cm} \huge{#1} \\ \vspace{5cm} \normalsize{End of placeholder}}}

% Allow the thesis to have bib in each chapter
\usepackage[sectionbib]{chapterbib}

% Hide the section in toc
\newcommand{\nocontentsline}[3]{}
\newcommand{\tocless}[2]{\bgroup\let\addcontentsline=\nocontentsline#1{#2}\egroup}

% environment for abstracts
\newenvironment{abstract}{\section*{Abstract}\it}{}

\addto\captionsenglish{%this line is required when using the babel package
  \renewcommand\bibname{References} % change Bibliography to references
}

%\newcommand{\pageplaceholder}[1]{\begin{center}\vspace*{\stretch{1}}\rotatebox{45}{\Huge{\textcolor{red}{Placeholder: #1}}}\vspace*{\stretch{1}}\end{center}}
%\newcommand{\pageplaceholder}[1]{\begin{center}\rotatebox{45}{\Huge{\textcolor{red}{Placeholder: #1}}}\end{center}}

%\usepackage{bitfoot} %make wonders to 
\usepackage{catchfilebetweentags}
\usepackage{wrapfig}

\usepackage{adjustbox} %Makes it possible to scale tables, if they are too big
\usepackage[final]{pdfpages}

\newcommand{\csubfloat}[2][]{%
  \makebox[0pt]{\subfloat[#1]{#2}}%
}

\usepackage{placeins} % Makes it possible to use floatbarriers = all declaired float before this point has to be printed

\usepackage{rotating} %Make it possible to rotate tables etc.
\usepackage{lscape}
\usepackage{bibentry} %Print bib entry in txt
\usepackage{nicefrac}

\usepackage{longtable,tabu}

\usepackage{afterpage} %Make it possible to place figure on odd/even page

\usepackage{multicol}

\usepackage{float}  % For using "H" to position figures

\usepackage{enumitem} % Make custom enumerations

\usepackage{epigraph} % To make quotes (https://tex.stackexchange.com/questions/53377/inspirational-quote-at-start-of-chapter)

\usepackage{csquotes} % To make different types of quotes (https://da.sharelatex.com/learn/Typesetting_quotations)

\usepackage[binary-units=true,separate-uncertainty=true,multi-part-units=single]{siunitx}	% Write SI units properly

%\usepackage[a4,pdflatex,center, cam]{crop}

% For equations:
\usepackage{dsfont}								% For math symbols
\DeclareMathAlphabet{\mathsfit}{\encodingdefault}{\sfdefault}{m}{sl}
\SetMathAlphabet{\mathsfit}{bold}{\encodingdefault}{\sfdefault}{bx}{n}
\newcommand{\tens}[1]{\bm{\mathsfit{#1}}}
\def\tA{{\tens{A}}}
\newcommand\numberthis{\addtocounter{equation}{1}\tag{\theequation}}  % To number equations written using \begin{align*}

% For theorem:
\newtheorem{theorem}{Theorem}
\newtheorem{corollary}{Corollary}
% \newenvironment{proof}{\begin{IEEEproof}}{\end{IEEEproof}}
\usepackage{algorithm}
\usepackage{algpseudocode}

% % %
%Bruges til at resette chapter couter ved ny part
%\makeatletter
%\@addtoreset{chapter}{part}
%\makeatother  
% % %

% Acronyms and glossaries
%\usepackage{etoolbox}
\usepackage[acronym,nomain,shortcuts]{glossaries} % [nonumberlist]
\renewcommand{\glossarysection}[2][]{} % Hack to remove titles from glossary lists
%\makeatletter
%
%\appto\newacronymhook{%
%  \newbool{glo@\the\glslabeltok @usedonlyonce}% define an additional switch per acronym
%}
%
%\patchcmd{\@gls@}{%
%  \glsunset{#2}%
%}{% write appropriate information to the main auxiliary file
%  \ifglsused{#2}{%
%    \write\@auxout{\global\setbool{glo@#2@usedonlyonce}{false}}%
%  }{%
%    \write\@auxout{\global\setbool{glo@#2@usedonlyonce}{true}}%
%  }%
%  \glsunset{#2}%
%}{}{}
%
%\patchcmd{\@gls@}{%
%  \glsentryfirst{#2}%
%}{% print the long form of the acronym if the acronym is used only once
%  \ifbool{glo@#2@usedonlyonce}{\glsentrylong{#2}}{\glsentryfirst{#2}}%
%}{}{}
%
%\makeatother
\makenoidxglossaries
%\newacronym{test}{TEST}{TEST}

\usepackage{lipsum}

%% Matlab2tikz
\usepackage{pgfplots}
  \pgfplotsset{width=.83\textwidth,compat=newest}
  %% the following commands are needed for some matlab2tikz features
  \usetikzlibrary{plotmarks}
  \usetikzlibrary{arrows.meta}
  \usepgfplotslibrary{patchplots}
  \usepackage{grffile}
  
\newenvironment{horizontallegend}[1][]{%
    \begingroup
    % inits/clears the lists (which might be populated from previous
    % axes):
    \pgfplots@init@cleared@structures
    \pgfplotsset{#1}%
}{%
    % draws the legend:
    \pgfplots@createlegend
    \endgroup
}%


%% Collection of papers
%\usepackage{collect}
%\definecollection{papers}
%\newcounter{PapersCounter}
%\makeatletter
%\newenvironment{paper}[3]
%  {\@nameuse{collect}{papers}{}
%    {}
%    {\clearpage \stepcounter{PapersCounter} \chapter*{Paper \thePapersCounter}\addcontentsline{toc}{section}{Paper \thePapersCounter : #1}\textbf{#1}\\\textit{#2}\\ \vspace{10 mm}\\#3\newpage}
%    {}%
%  }
%{\@nameuse{endcollect}}
%\makeatother

\usepackage{bookmark}
\usepackage{fontawesome5} % font awesome                                                                                                            
                                                                                                            \usepackage{textcomp}
                                                                                                            \usepackage{pifont}

                                                                                                            \usepackage{adjustbox}

% Collection of papers
\usepackage{collect}
\definecollection{papers}
\newcounter{PapersCounter}
\makeatletter
\newenvironment{paper}[6]
{\@nameuse{collect}{papers}{}
    {}
    {\clearpage \stepcounter{PapersCounter} \paperlabel{#1}{\thePapersCounter} \chapter*{Paper \thePapersCounter}\addcontentsline{toc}{section}{Paper \thePapersCounter : #2}\textbf{#2}\\\textit{#3}\\ \vspace{10 mm}\\#4 \vspace{10 mm}\\#5 \vspace{0 mm}\\\textit{#6} \newpage}
    {}%
}
{\@nameuse{endcollect}}
\makeatother

%

% Paper referencing
\makeatletter
\newcommand{\paperlabel}[2]{%
    \protected@write \@auxout {}{\string \newlabel {#1}{{Paper #2}{\thepage}{#2}{#1}{}} }%
    \hypertarget{#1}{}
}
\makeatother



% For subgrous in tables
\newcommand{\rowgroup}[1]{\hspace{-1em}#1}

% For centering in fixed width columnds 
\newcolumntype{C}[1]{>{\centering\arraybackslash}p{#1}}

\usepackage{tabularx}

\makeatletter % Redefinition of Description List Items source: https://tex.stackexchange.com/a/1248/13552
\let\orgdescriptionlabel\descriptionlabel
\renewcommand*{\descriptionlabel}[1]{%
  \let\orglabel\label
  \let\label\@gobble
  \phantomsection
  \edef\@currentlabel{#1\unskip}%
  %\edef\@currentlabelname{#1}%
  \let\label\orglabel
  \orgdescriptionlabel{#1}%
}
                                                                                                            \makeatother

                                                                                                            \usepackage[abbreviations]{foreign}  % For \ie, \eg \etc
                                                                                                            \newboolean{showcomments}
\setboolean{showcomments}{true}
\ifthenelse{\boolean{showcomments}}
{ \newcommand{\mynote}[3]{
   \fbox{\bfseries\sffamily\scriptsize#1}
   {\small$\blacktriangleright$\textsf{\emph{\color{#3}{#2}}}$\blacktriangleleft$}}}
{ \newcommand{\mynote}[3]{}}
\newcommand{\vs}[1]{\mynote{Valerio}{#1}{blue}}
\newcommand{\pf}[1]{\mynote{Pascal}{#1}{pink}}
\newcommand{\lar}[1]{\mynote{Lars}{#1}{red}}
\newcommand{\dl}[1]{\mynote{Daniel}{#1}{green}}
\newcommand{\db}[1]{\mynote{Dorian}{#1}{magenta}}
\newcommand{\sys}{\textsc{MinervaFS}\xspace}